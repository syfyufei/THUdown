
\hypertarget{ux5f15ux8a00}{%
\chapter{引言}\label{ux5f15ux8a00}}

Scroll down for the English version of README.

本文是R语言软件包\texttt{THUDown}的使用手册,从架构、安装方法和使用方法三个方面对\texttt{THUDown}进行了介绍。\footnote{此文撰写的得到了清华大学赵赫和成功两位同学的支持,其中赵赫对``使用方法''章节的内容提供了主要的支持,成功同学对``安装方法''章节提供了支持。}

\texttt{THUDown}是\texttt{Tsinghua\ University}和\texttt{markdown}的缩写,它是一个基于
\LaTeX 和
Quarto开发的清华大学学位论文Quarto模板,包括本科综合论文训练、硕士论文、博士论文以及博士后出站报告。它旨在让用户能够使用语法简单的类markdown文档(Quarto)来实现调用
\LaTeX 进行复杂的文件编译。\texttt{THUDown}亦被封装在\href{https://sammo3182.github.io/software/drhutools/}{\texttt{drhutools}}软件包中,用户可借助\texttt{drhutools::THUDown()}进行下载使用。

\textbf{需要特别强调的是},本软件包是基于\href{https://github.com/tuna/thuthesis}{ThuThesis}开发的Quarto版本,因此其亦可被视作其拓展包。
\texttt{THUDown}的\textbf{主要}
\LaTeX 代码来自于\textbf{ThuThesis}宏包开发者的贡献,\texttt{THUDown}在\href{https://www.latex-project.org/lppl/lppl-1-3c/}{LaTeX项目公共许可证
v1.3c}的授权使用范围内使用\textbf{ThuThesis}宏包的内容。

请在\href{https://www.latex-project.org/lppl/lppl-1-3c/}{LaTeX项目公共许可证
v1.3c}的使用许可范围内使用\texttt{THUDown}。

\hypertarget{ux67b6ux6784}{%
\section{架构}\label{ux67b6ux6784}}

\texttt{THUdown}
的编译框架采用了一个分层和集成的方法来生成学术文档。其核心编译流程可以概括为以下几个阶段:

\begin{enumerate}
\def\labelenumi{\arabic{enumi}.}
\item
  \textbf{内容创建与转换}: 初始内容以 Quarto
  文件格式撰写。这种格式主要针对文本的表示和结构化,使其便于编辑和版本控制。
\item
  \textbf{中间表示}: 利用 \texttt{pandoc} 这一强大的文档转换工具,Quarto
  文件被转化为 \LaTeX 格式。在这一阶段,文本和元数据被转化为
  \LaTeX 指令和环境,为下一步套用模板做准备。
\item
  \textbf{模板应用}: 转换得到的 \LaTeX 文件被整合到 \texttt{ThuThesis}
  模板中。\texttt{ThuThesis} 是为清华大学学术论文设计的
  \LaTeX 模板,能确保文档满足清华大学的格式要求。
\item
  \textbf{文档编译}: 最后,通过使用 \texttt{make} 命令(该命令引导
  \LaTeX 编译器),将 \LaTeX 文件编译成为最终的 PDF 格式的学术文档。
\end{enumerate}

这一架构确保了从内容创建到最终文档输出的整个过程都是高度自动化的,同时也保证了文档质量和格式的一致性。


\hypertarget{ux4e0bux8f7d}{%
\chapter{下载}\label{ux4e0bux8f7d}}

\texttt{THUDown}是一个基于 \LaTeX 和
Quarto开发的清华大学学位论文Quarto模板,因此其使用需要下载\texttt{THUDown}核心软件包,亦需要恰当安装
\LaTeX 、Quarto和ThuThesis等支持项目

\hypertarget{ux6838ux5fc3ux8f6fux4ef6ux5305}{%
\section{核心软件包}\label{ux6838ux5fc3ux8f6fux4ef6ux5305}}

本节提供了\textbf{开发版}和\textbf{发布版}下载路径。

\begin{itemize}
\item
  开发版

  \begin{itemize}
  \tightlist
  \item
    \href{https://github.com/syfyufei}{GitHub Releases}
  \end{itemize}
\item
  发布版

  \begin{itemize}
  \item
    \href{}{CRAN}
  \item
    \href{https://sammo3182.github.io/software/drhutools/}{\texttt{drhutools}}
  \end{itemize}
\end{itemize}

\hypertarget{ux652fux6301ux9879ux76ee}{%
\section{支持项目}\label{ux652fux6301ux9879ux76ee}}

\hypertarget{ux53d1ux884cux7248}{%
\subsection{\texorpdfstring{\LaTeX 发行版}{发行版}}\label{ux53d1ux884cux7248}}

本文提供两种不同的 \LaTeX 发行版选项,用户可以根据自身需求选择安装

\begin{itemize}
\item
  \textbf{\texttt{TinyTeX}}。TinyTeX是一个瘦身版的TeX
  Live,用户可以参考其主页的\href{https://yihui.org/tinytex/}{支持文档}自行安装。需要强调的是,因为\texttt{THUDown}和ThuThesis中包含大量宏包的调用,因此安装\texttt{TinyTeX}的用户在享受其便捷的同时,亦需要在安装第三方宏包上伤脑筋,尤其是无法通过\texttt{tlmgr}安装的宏包。对于这些包,请参见\href{https://github.com/rstudio/tinytex/issues/126\#issuecomment-503020154}{此issue}
\item
  \textbf{\texttt{TexLive}}。用户亦可通过\href{https://tug.org/texlive/}{Texlive主页}安装完整版的\texttt{TexLive}。
\end{itemize}

\hypertarget{quarto}{%
\subsection{Quarto}\label{quarto}}

Quarto是一套开源的科技出版系统。
它使得用户能够使用Jupyter、Python、R、Julia和Observable创作HTML、PDF、MS
Word、ePub等格式的可复制、高质量的文章、演示文稿、网站、博客和书籍。
Quarto是\texttt{THUDown}必须项,请参考\href{https://quarto.org/docs/get-started/}{Quarto支持页面}进行安装。

\hypertarget{thuthesis}{%
\subsection{ThuThesis}\label{thuthesis}}

\href{https://github.com/tuna/thuthesis}{ThuThesis}一个简单易用的清华大学学位论文
LaTeX 模板,\texttt{THUDown}的主体
\LaTeX 代码来自ThuThesis。请参考\href{https://github.com/tuna/thuthesis\#readme}{ThuThesis的Github项目主页}进行安装。


\hypertarget{ux4f7fux7528ux8bf4ux660e}{%
\chapter{使用说明}\label{ux4f7fux7528ux8bf4ux660e}}

本章将从编辑和编译两个部分对\texttt{THUdown}的使用方法进行详细的说明。

\hypertarget{ux7f16ux8f91}{%
\section{编辑}\label{ux7f16ux8f91}}

\hypertarget{ux4e3bux4f53ux67b6ux6784}{%
\subsection{主体架构}\label{ux4e3bux4f53ux67b6ux6784}}

├── abstract.tex

├── acknowledgements.tex

├── appendix-survey.tex

├── appendix-translation.tex

├── appendix.tex

├── comments.tex

├── committee.tex

├── denotation.tex

├── main.tex

├── /quarto\_file

├── resolution.tex

└── resume.tex

\hypertarget{ux5bf9quarto.qmdux6587ux6863ux7684ux7f16ux8f91}{%
\subsection{对Quarto(.qmd)文档的编辑}\label{ux5bf9quarto.qmdux6587ux6863ux7684ux7f16ux8f91}}

data/quarto\_file文件夹

正文部分

具体功能的使用

附录部分

\hypertarget{ux5bf9latex.texux6587ux6863ux7684ux7f16ux8f91}{%
\subsection{对Latex(.tex)文档的编辑}\label{ux5bf9latex.texux6587ux6863ux7684ux7f16ux8f91}}

data文件夹中的其他文件

\hypertarget{ux7f16ux8bd1}{%
\section{编译}\label{ux7f16ux8bd1}}

\hypertarget{ux7f16ux8bd1ux65b9ux6cd5}{%
\subsection{编译方法}\label{ux7f16ux8bd1ux65b9ux6cd5}}

\begin{enumerate}
\def\labelenumi{\arabic{enumi}.}
\item
  确保你的系统已经安装了\texttt{zsh},\texttt{quarto},\texttt{awk},和\texttt{make}这些工具。
\item
  打开终端,导航(\texttt{cd})到\texttt{knit.sh}文件所在的目录,即\texttt{THUdown}的根目录。
\item
  运行以下命令使\texttt{knit.sh}文件变为可执行:
\end{enumerate}

\texttt{\{bash\ eval=FALSE,\ include=FALSE\}\ chmod\ +x\ knit.sh}

\begin{enumerate}
\def\labelenumi{\arabic{enumi}.}
\setcounter{enumi}{3}
\tightlist
\item
  执行脚本来编译你的\texttt{THUdown}包:
\end{enumerate}

\texttt{\{bash\ eval=FALSE,\ include=FALSE\}\ ./knit.sh}

\begin{enumerate}
\def\labelenumi{\arabic{enumi}.}
\setcounter{enumi}{4}
\tightlist
\item
  脚本会按照上述的编译原理进行操作,并最终生成你的论文。
\end{enumerate}


