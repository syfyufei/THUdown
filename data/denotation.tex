% !TeX root = ../thuthesis-example.tex

\begin{denotation}[3cm]
  \item[CCP] 中国共产党(Chinese Communist Party)
  \item[CPC] 中国共产党(Chinese Communist Party)[注:CCP和CPC都可以指中国共产党]
  \item[NPC] 全国人民代表大会(National People's Congress)
  \item[CPPCC] 全国政协(Chinese People's Political Consultative Conference)
  \item[PLA] 中国人民解放军(People's Liberation Army)
  \item[CASS] 中国社会科学院(Chinese Academy of Social Sciences)
  \item[CISA] 中国国际战略学会(China Institute of International Strategic Studies)
  \item[RF] 改革派(Reformist Faction)
  \item[CF] 保守派(Conservative Faction)
  \item[FTZ] 自贸区(Free Trade Zone)
  \item[BRI] 一带一路(Belt and Road Initiative)
  \item[SOE] 国有企业(State-owned Enterprises)
  \item[CMI] 中央军委(Central Military Commission)
  \item[PBOC] 中国人民银行(People's Bank of China)
  \item[CAC] 网络安全和信息化委员会(Cyberspace Administration of China)
  \item[SAR] 特别行政区(Special Administrative Region)
  \item[CFA] 中央财经委员会(Central Finance and Economics Commission)
  \item[HKSAR] 香港特别行政区(Hong Kong Special Administrative Region)
  \item[MSAR] 澳门特别行政区(Macao Special Administrative Region)
  \item[TAR] 西藏自治区(Tibet Autonomous Region)
  \item[XUAR] 新疆维吾尔自治区(Xinjiang Uyghur Autonomous Region)
  \item[CACG] 中国反腐败工作领导小组(China Anti-Corruption Coordination Group)
  \item[RMB] 人民币(Renminbi)
\end{denotation}




% 也可以使用 nomencl 宏包,需要在导言区
% \usepackage{nomencl}
% \makenomenclature

% 在这里输出符号说明
% \printnomenclature[3cm]

% 在正文中的任意为都可以标题
% \nomenclature{PI}{聚酰亚胺}
% \nomenclature{MPI}{聚酰亚胺模型化合物,N-苯基邻苯酰亚胺}
% \nomenclature{PBI}{聚苯并咪唑}
% \nomenclature{MPBI}{聚苯并咪唑模型化合物,N-苯基苯并咪唑}
% \nomenclature{PY}{聚吡咙}
% \nomenclature{PMDA-BDA}{均苯四酸二酐与联苯四胺合成的聚吡咙薄膜}
% \nomenclature{MPY}{聚吡咙模型化合物}
% \nomenclature{As-PPT}{聚苯基不对称三嗪}
% \nomenclature{MAsPPT}{聚苯基不对称三嗪单模型化合物,3,5,6-三苯基-1,2,4-三嗪}
% \nomenclature{DMAsPPT}{聚苯基不对称三嗪双模型化合物(水解实验模型化合物)}
% \nomenclature{S-PPT}{聚苯基对称三嗪}
% \nomenclature{MSPPT}{聚苯基对称三嗪模型化合物,2,4,6-三苯基-1,3,5-三嗪}
% \nomenclature{PPQ}{聚苯基喹噁啉}
% \nomenclature{MPPQ}{聚苯基喹噁啉模型化合物,3,4-二苯基苯并二嗪}
% \nomenclature{HMPI}{聚酰亚胺模型化合物的质子化产物}
% \nomenclature{HMPY}{聚吡咙模型化合物的质子化产物}
% \nomenclature{HMPBI}{聚苯并咪唑模型化合物的质子化产物}
% \nomenclature{HMAsPPT}{聚苯基不对称三嗪模型化合物的质子化产物}
% \nomenclature{HMSPPT}{聚苯基对称三嗪模型化合物的质子化产物}
% \nomenclature{HMPPQ}{聚苯基喹噁啉模型化合物的质子化产物}
% \nomenclature{PDT}{热分解温度}
% \nomenclature{HPLC}{高效液相色谱(High Performance Liquid Chromatography)}
% \nomenclature{HPCE}{高效毛细管电泳色谱(High Performance Capillary lectrophoresis)}
% \nomenclature{LC-MS}{液相色谱-质谱联用(Liquid chromatography-Mass Spectrum)}
% \nomenclature{TIC}{总离子浓度(Total Ion Content)}
% \nomenclature{\textit{ab initio}}{基于第一原理的量子化学计算方法,常称从头算法}
% \nomenclature{DFT}{密度泛函理论(Density Functional Theory)}
% \nomenclature{$E_a$}{化学反应的活化能(Activation Energy)}
% \nomenclature{ZPE}{零点振动能(Zero Vibration Energy)}
% \nomenclature{PES}{势能面(Potential Energy Surface)}
% \nomenclature{TS}{过渡态(Transition State)}
% \nomenclature{TST}{过渡态理论(Transition State Theory)}
% \nomenclature{$\increment G^\neq$}{活化自由能(Activation Free Energy)}
% \nomenclature{$\kappa$}{传输系数(Transmission Coefficient)}
% \nomenclature{IRC}{内禀反应坐标(Intrinsic Reaction Coordinates)}
% \nomenclature{$\nu_i$}{虚频(Imaginary Frequency)}
% \nomenclature{ONIOM}{分层算法(Our own N-layered Integrated molecular Orbital and molecular Mechanics)}
% \nomenclature{SCF}{自洽场(Self-Consistent Field)}
% \nomenclature{SCRF}{自洽反应场(Self-Consistent Reaction Field)}
